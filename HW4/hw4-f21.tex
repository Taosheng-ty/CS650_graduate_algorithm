\documentclass[addpoints, 11pt]{exam}
\usepackage{url}
\usepackage{amsmath,amsthm,enumitem}
\usepackage{graphicx}
%\input myfonts

\newtheorem*{claim}{Claim}
\title{CS 6150: HW4 -- Graphs, Randomized algorithms}
\date{Submission date: Wednesday, Nov 10, 2021 (11:59 PM)}
\begin{document}
\maketitle
\begin{center}
\fbox{\fbox{\parbox{5.5in}{\centering
This assignment has \numquestions\ questions, for a total of \numpoints\
points. % and \numbonuspoints\ bonus points.  
Unless otherwise specified, complete and reasoned arguments will be expected for all answers. }}}
\end{center}

\qformat{Question \thequestion: \thequestiontitle\dotfill \textbf{[\totalpoints]}}
\pointname{}
\bonuspointname{}
\pointformat{[\bfseries\thepoints]}

\begin{center}
  \gradetable
\end{center}
\newpage
\paragraph{Instructions.}  For all problems in which you are asked to develop an algorithm, write down the pseudocode, along with a rough argument for correctness and an analysis of the running time (unless specified otherwise). Failure to do this may result in a penalty. If you are unsure how much detail to provide, please contact the instructors on Piazza.

\begin{questions}
\titledquestion{QuickSelect}[6]
Recall that given an (unsorted) array of {\bf distinct} integers $A[0, 1, \dots, n-1]$ and a parameter $1 \le k \le n$, the Selection problem asks to  find the $k$th smallest entry of $A$. In class, we saw an algorithm that used a randomized implementation of ApproximateMedian, and showed that it leads to an $O(n)$ time algorithm.  Let us now consider a different procedure, that is similar to QuickSort.

\textsc{Procedure QuickSelect}$(A, k)$
\begin{enumerate}
\item If $|A|=1$, return the only element
\item Select $x$ from $A$ uniformly at random
\item Form arrays $B$ and $C$, containing the elements of $A$ that are $<x$ and $>x$ respectively
\item If $|B| = (k-1)$, return $x$, else if $|B| < (k-1)$, return \textsc{QuickSelect}$(C, k-|B|-1)$, else return \textsc{QuickSelect}$(B, k)$
\end{enumerate}

Let $T(n)$ be defined as the {\bf expected running time} of QuickSelect on an array of length $n$. Using the law of conditional expectation, prove that
\[ T(n) \le n + \sum_{j=1}^n \frac{1}{n} \max \{ T(j-1), T(n-j)\}. \]
Using this along with $T(1) = 1$, prove that $T(n) \le 4n$.  Write down a description of all the events you use when you use conditional expectation.

(For the purposes of this question, you may ignore the additional $O(1)$ time for steps (1-2) and (4) of the procedure above.) [{\em Hint:} Follow the analysis for QuickSort seen in class, use induction.]

{\bf Side note.} It is interesting to see that the constant term (the 4 in $4n$) above is much better than what we had for the deterministic algorithm we saw before. It turns out that there's a way of improving the constant further: instead of choosing $x$ uniformly at random, we pick a small sample from the array and pick the sample median. 

\titledquestion{Sampling from a stream}[6]
If you have an array of $n$ elements, sampling one at random is easy: you choose an index $i$ at random in $\{0,1, \dots, n-1\}$ and return the $i$th element. Now suppose you have a {\em stream} of elements $a_1, a_2, \dots$ (suppose they are \underline{all distinct} for simplicity), and you don't know how many will arrive beforehand. Your goal is the following: at the end of the stream, you should output a random element from the stream. 

The trivial algorithm is to store all the elements in an array (say a dynamic array), and in the end, output a random element. But it turns out that this can be done with very little memory. 

Consider the following procedure: we maintain a special variable $x$, initialized to the first element of the array. At time $t$, upon seeing $a_t$, we set $x = a_t$ with probability $1/t$, otherwise we keep $x$ unchanged. 

Prove that in the end, the variable $x$ stores a uniformly random sample from the stream. (In other words, if the stream had $N$ elements, $\Pr[ x= a_i] = 1/N$ for all $i$.)

[{\em Hint:} try doing a direct computation.]

\titledquestion{Walking on a path}
Consider a path of length $n$ comprising vertices $v_0, v_1, \dots, v_n$.  A particle starts at $v_0$ at $t=0$, and in each time step, it moves to a {\bf uniformly random neighbor} of the current vertex. Thus if it is at $v_s$ at time $t$ for some $s>0$, then at time $(t+1)$, it moves to $v_{s+1}$ or $v_{s-1}$ with probability $1/2$ each. (If it is at $v_0$, the only neighbor is $v_1$ and so it moves there.) The particle gets ``absorbed'' once it reaches $v_n$ and the walk stops.

Define $T(i)$ as the expected number of time steps taken by a particle {\em starting at} $i$ to reach $v_n$. By definition, $T(n) =0$. 
\begin{parts}
\part[5] Prove that $T(0) = 1+T(1)$, and further, that for any $0 < s< n$, $T(s) = 1 + \frac{T(s-1) + T(s+1)}{2}$. 
\part[5] Use this to prove that $T(s) = (2s+1) + T(s+1)$ for all $0 \le s < n$, and then find a closed form for $T(0)$.  [{\em Hint: } Use induction.]
\part[2] Give an upper bound for the probability that the particle walks for $> 4n^2$ steps without getting absorbed.
\end{parts}

\titledquestion{Birthdays and applications}
Suppose we have $n$ people, each of whom has their birthday on a random day of the year. Suppose {\bf there are $m$ days in a year}, and let us pretend that this is some parameter. 

\begin{parts}
\part[5] What is the expected {\em number of pairs} $(i, j)$ with $i < j$ such that person $i$ and person $j$ have the same birthday? For what value of $n$ (as a function of $m$) does this number become $1$?
\part[7] This idea has some nice applications in CS, one of which is in estimating the ``support'' of a distribution. Suppose we have a radio station that claims to have a library of one million songs, and suppose that the radio station plays these songs by picking, at each step a uniformly random song from its library (with replacement), playing it, then picking the next song, and so on.

Suppose we have a listener who started listening when the station began, and noticed that among the first 200 songs, there was a repetition (i.e., a song played twice). Prove that the probability of this happening (conditioned on the library size being a million songs) is $<0.05$.  Note that this gives us ``reasonable doubt'' about the station's claim that its library has a million songs.

{\em Hint: } Compute the probability of the complementary event ---that all songs would be distinct--- and prove that it must be large. You may use the inequality $(1-x)^n \ge 1-nx$ (for $x > 0$ and a positive integer $n$) without proof. 

[This idea has many applications in CS, for estimating the size of sets without actually enumerating them.]
\end{parts} 

\titledquestion{Checking matrix multiplication}
Matrix multiplication is one of the classic algorithmic problems.  Consider the problem of multiplying two $n \times n$ matrices over the field $\mathbf{F}_2$ (i.e., we have matrices with entries $0/1$, and we perform all computations modulo $2$; e.g., 0*0 + 1*1 + 1*1 + 1*0 = 0).

The best known algorithms here are messy and take time $O(n^{2.36...})$.  However, the point of this exercise is to prove a simpler statement. Suppose someone gives a matrix $C$ and claims that $C = AB$, can we {\em quickly verify} if the claim is true?

\begin{parts}
\part[5] First prove a warm-up statement: suppose $a$ and $b$ are any two $0/1$ vectors of length $n$, and suppose that $a \ne b$.  Then, for a random binary vector $x \in \{0,1\}^n$ (one in which each coordinate is chosen uniformly at random), prove that $\Pr [ \langle a, x\rangle \ne \langle b, x\rangle~ (\text{mod } 2)] = 1/2$.  [In other words, with a probability $1/2$, we can ``catch'' the fact that $a \ne b$.]
\part[6] Now, design an $O(n^2)$ time algorithm that tests if $C = AB$ and has a success probability $\ge 1/2$. (You need to bound both the running time and probability.)
\part[3] Show how to improve the success probability to $7/8$ while still having running time $O(n^2)$.
\end{parts}

\end{questions}
\end{document}